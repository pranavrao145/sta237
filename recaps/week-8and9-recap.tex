\documentclass[12pt]{article}
\usepackage{amsmath}
\usepackage{amsfonts}
\usepackage{amssymb}
\usepackage{amsthm}
\usepackage{pifont}
\usepackage{fancyhdr}
\usepackage{hyperref}
\usepackage{float,graphicx}
\usepackage{framed}
\usepackage[margin=3cm, headheight=15pt]{geometry}


\pagestyle{fancyplain}
\fancypagestyle{plain}{
	\renewcommand{\headrulewidth}{0.4pt}
}


\lhead{\fancyplain{Pranav Rao}{Pranav Rao}}
\rhead{\fancyplain{Week 8 and 9 Recap}{Week 8 and 9 Recap}}

\title{Week 8 and 9 Recap}
\author{Pranav Rao}
\date{November 11, 2023}

\begin{document}
\maketitle

\section{Joint Distributions}
\begin{itemize}
	\item Given two random variables defined on the same
	      probability space, the \textbf{joint probability distribution} is the
	      corresponding probability distribution on all possible pairs of
	      outputs.
	\item A joint probability distribution ``encodes'' two other types of probability distributions:
	      \begin{itemize}
		      \item \textbf{Marginal distributions:} the distributions of each individual random variable.
		      \item \textbf{Conditional distributions:} deal with how the
		            outputs of one random variable are distributed when
		            given information about the other random variable.
	      \end{itemize}
	\item We use the \textbf{covariance} to measure the relationship between two random variables (definition depends on types of random variables, see below).
	      \begin{itemize}
		      \item However, the covariance is in the same units as the
		            variances, and therefore is affected by the units that we
		            choose to measure the data.
		      \item So, we also define the \textbf{correlation}, which is unitless (definition depends on types of random variables, see below)
		            \begin{itemize}
			            \item The correlation has values on $[-1, 1]$:
			                  \begin{itemize}
				                  \item A correlation of $0$ indicates there is
				                        no linear relationship between the two
				                        random variables (i.e. they are
				                        \emph{linearly independent}).
				                  \item A correlation of $\pm 1$ indicates a
				                        perfect linear positive or negative
				                        relationship.
				                  \item Correlation values near $\pm 1$
				                        imply a ``strong'' correlation, values near $0$ imply a
				                        ``weak'' correlation, and values in between imply
				                        ``moderate'' correlation.
			                  \end{itemize}
		            \end{itemize}
	      \end{itemize}
	\item Like regular random variables/distributions, there are two types of
	      joint distributions, which we will look into in separate sections of
	      this recap:
	      \begin{itemize}
		      \item Discrete
		      \item Continuous
	      \end{itemize}
\end{itemize}

\subsection{Joint Distributions of Discrete Random Variables}

\begin{itemize}
	\item Consider two discrete random variables $X$ and $Y$.
	\item The \textbf{joint probability mass function}, defined as $P_{XY}(X=x,
		      Y=y)$ is normally given in a table, and must satisfy two properties for
	      every $x$, $y$:
	      \begin{itemize}
		      \item $0 \leq P_{XY}(x, y) \leq 1$
		      \item $\displaystyle\sum_{x} \displaystyle\sum_{y} P_{XY}(x, y) = 1$
	      \end{itemize}
	\item The \textbf{marginal probability mass functions} $P_X(X=x), P_Y(Y=y)$ can be defined as such:
	      \begin{itemize}
		      \item $P_X(X=x) = \displaystyle\sum_{y} P_{XY}(x, y)$
		      \item $P_Y(Y=y) = \displaystyle\sum_{x} P_{XY}(x, y)$
	      \end{itemize}

	\item The \textbf{joint cumulative distribution function} $F_{XY}(x, y)$ of
	      two discrete random variables $X, Y$ with PMF $P_{XY}(x, y)$ can be defined as such:
	      \[
		      F_{XY}(x, y) = P_{XY}(X \leq x, Y \leq y)
	      \]
	\item The \textbf{marginal cumulative distribution functions} $F_X(x)$ and $F_Y(y)$ can be defined as such:
	      \begin{itemize}
		      \item $F_X(x) = \lim_{y\to\infty}F_{XY}(x, y) = F_{XY} (x, \infty)$ (shorthand) for any $x$
		      \item $F_Y(y) = \lim_{x\to\infty}F_{XY}(x, y) = F_{XY} (\infty, y)$ (shorthand) for any $y$
	      \end{itemize}
	\item \textbf{Conditional probability mass functions} require more detail and are discussed in a later section.
	\item The \textbf{expected value} can be calculated as $E[g(X, Y)] = \displaystyle\sum_{x}
		      \displaystyle\sum_{y} g(x, y) \cdot F_{XY}(x, y)$, where $g(X, Y)$ is a
	      function of $X, Y$ and $F_{XY}(x, y)$ is a joint PMF of $X$ and $Y$. Some consequences of this are:
	      \begin{itemize}
		      \item When $g(X, Y) = X$, $E(g(X, Y)) = E(X) = \displaystyle\sum_{x} = x P_X(x)$, where $P_X(x)$ is the marginal pmf of $X$
		      \item When $g(X, Y) = Y$, $E(g(X, Y)) = E(Y) = \displaystyle\sum_{y} = y P_Y(y)$, where $P_Y(y)$ is the marginal pmf of $Y$
	      \end{itemize}
	\item The \textbf{variance} is still defined as $Var(g(X, Y)) = E[g(X,
				      Y)^2] - (E[g(X, Y)])^2$, where $g(X, Y)$ is a function of $X, Y$. Some
	      consequences of this are:
	      \begin{itemize}
		      \item $Var(X) = \displaystyle\sum_{x} x^2 P_(X) - (E(X))^2$, where $P_X(x)$ is the marginal pmf of $X$
		      \item $Var(Y) = \displaystyle\sum_{y} y^2 P_(Y) - (E(Y))^2$, where $P_Y(y)$ is the marginal pmf of $Y$
	      \end{itemize}
	\item As usual, the \textbf{standard deviation} $\sigma_X, \sigma_Y$,
	      can be calculated by taking the square roots of the respective
	      variances.
	\item The \textbf{covariance} $Cov(X, Y)$ of $X$ and $Y$ in a joint discrete distribution can be calculated as such:
	      \[
		      Cov(X, Y) = \displaystyle\sum_{x} \displaystyle\sum_{y} xy P_{XY}(x, y) - E(X)E(Y)
	      \]
	\item The \textbf{correlation} $\rho_{XY}$ of $X$ and $Y$ in a joint discrete distribution can be calculated as such:
	      \[
		      \rho_{XY} = \frac{Cov(X, Y)}{\sigma_X \sigma_Y}
	      \]
	\item Discrete random variables are \textbf{independent} if their joint PMF factors into a product of the marginal PMFs.
\end{itemize}

\subsection{Joint Distributions of Continuous Random Variables}

\begin{itemize}
	\item Consider two continuous random variables $X$ and $Y$.
	\item The \textbf{joint probability distribution function}, defined as a piecewise
	      continuous function $f_{XY}(x, y)$ or sometimes just $f(x, y)$, must satisfy two properties for
	      every $x$, $y$ in the domain of $f_{XY}$:
	      \begin{itemize}
		      \item $f_{XY}(x, y) \geq 0$
		      \item $\int_{y} \int_{x} f_{XY}(x, y) dx dy = 1$ (note that $\int_{x}$
		            means to integrate over all $x$ from $(-\infty, \infty)$),
		            same for every other reference in this document
	      \end{itemize}
	\item The \textbf{marginal probability distribution functions} $f_X(x), f_Y(y)$ can be defined as such:
	      \begin{itemize}
		      \item $f_X(x) = \int_{y} f_{XY}(x, y)dy$
		      \item $f_Y(y) = \int_{x} f_{XY}(x, y)dx$
	      \end{itemize}
	\item The \textbf{joint cumulative distribution function} $F_{XY}(a, b)$ of
	      two continuous random variables $X, Y$ with PDF $f_{XY}(x, y)$ can be defined as such:
	      \[
		      F_{XY}(a, b) = f_{XY}(X \leq a, Y \leq b) = \int_{-\infty}^{b} \int_{-\infty}^{a} f_{XY}(x, y) dx dy
	      \]
	      \begin{itemize}
		      \item It must satisfy some properties:
		            \begin{itemize}
			            \item $F_{XY}(\infty, \infty) = 1$
			            \item $F_{XY}(-\infty, y) = F_{XY}(x, -\infty) = 0$
		            \end{itemize}
	      \end{itemize}
	\item The \textbf{marginal cumulative distribution functions} $F_X(x)$ and $F_Y(y)$ can be defined as such:
	      \begin{itemize}
		      \item $F_X(x) = \lim_{y\to\infty}F_{XY}(x, y) = F_{XY} (x, \infty)$ (shorthand) for any $x$
		      \item $F_Y(y) = \lim_{x\to\infty}F_{XY}(x, y) = F_{XY} (\infty, y)$ (shorthand) for any $y$
	      \end{itemize}
	\item \textbf{Conditional probability distribution functions} require more detail and are discussed in a later section.
	\item The \textbf{expected value} can be calculated as $E[g(X, Y)] = \int_{y} \int_{x} g(x, y) f_{XY}(x, y) dx dy$, where $g(X, Y)$ is a
	      function of $X, Y$ and $f_{XY}(x, y)$ is a joint PDF of $X$ and $Y$. Some consequences of this are:
	      \begin{itemize}
		      \item When $g(X, Y) = X$, $E(g(X, Y)) = E(X) = \int_{x} x\cdot f_X(x)dx$, where $f_X(x)$ is the marginal pdf of $X$
		      \item When $g(X, Y) = Y$, $E(g(X, Y)) = E(Y) = \int_{y} y\cdot f_Y(y)dy$, where $f_Y(y)$ is the marginal pdf of $Y$
	      \end{itemize}
	\item The \textbf{variance} is still defined as $Var(g(X, Y)) = E[g(X,
				      Y)^2] - (E[g(X, Y)])^2$, where $g(X, Y)$ is a function of $X, Y$. Some
	      consequences of this are:
	      \begin{itemize}
		      \item $Var(X) = \int_{x} x^2 f_X(x)dx - (E(X))^2$, where $f_X(x)$ is the marginal pdf of $X$
		      \item $Var(Y) = \int_{y} y^2 f_Y(y)dy - (E(Y))^2$, where $f_Y(y)$ is the marginal pdf of $Y$
	      \end{itemize}
	\item As usual, the \textbf{standard deviation} $\sigma_X, \sigma_Y$,
	      can be calculated by taking the square roots of the respective
	      variances.
	\item The \textbf{covariance} $Cov(X, Y)$ of $X$ and $Y$ in a joint continuous distribution can be calculated as such:
	      \[
		      Cov(X, Y) = \int_{x} \int_{y} xy f_{XY} (x, y) dy dx - E(X)E(Y)
	      \]
	\item The \textbf{correlation} $\rho_{XY}$ of $X$ and $Y$ in a joint continuous distribution can be calculated as such:
	      \[
		      \rho_{XY} = \frac{Cov(X, Y)}{\sigma_X \sigma_Y}
	      \]
	\item Continuous random variables are \textbf{independent} if their joint PDF
	      factors into a product of the marginal PDFs.
\end{itemize}

\section{Conditional Probabilities of Joint Distributions}

\subsection{Discrete Distributions}
Note: for this section, whenever it says ``given $Y=y$'' (for any given random variable), you can also replace that with ``given an event $A$ has occurred.''

\begin{itemize}
	\item The \textbf{conditional probability mass functions} $P_{X | Y}(x|y)$ and $P_{Y | X}(y|x)$ of a discrete distribution can be defined as such (where $P(x, y)$ is a joint PMF for $X$ and $Y$):
	      \begin{itemize}
		      \item ``Conditional probability mass function of $X$ given that
		            $Y=y$'': $P_{X | Y}(x|y) = \frac{P(\{X=x\}\cap
				            \{Y=y\})}{P(Y=y)}=\frac{P(x,y)}{P_Y(y)}$, provided $P_Y(y) >
			            0$ (or else it doesn't exist).
		      \item ``Conditional probability mass function of $Y$ given that
		            $X=x$'': $P_{Y | X}(y|x) = \frac{P(\{Y=y\}\cap
				            \{X=x\})}{P(X=x)}=\frac{P(x,y)}{P_X(x)}$, provided $P_X(x) >
			            0$ (or else it doesn't exist).
	      \end{itemize}
	\item Below are some properties of conditional PMFs:
	      \begin{enumerate}
		      \item Conditional PMFs are valid PMFs, meaning that they satisfy the properties of a PMF (see definition of PMF for Joint Distribution above).
		      \item \emph{In general}, the conditional PMF of $X$ given $Y$ does not equal the conditional distribution of $Y$ given $X$, i.e.:
		            \[
			            P_{X|Y}(x | y) \neq  P_{Y|X}(y | x)
		            \]
		      \item If $X$ and $Y$ are independent, then:
		            \begin{itemize}
			            \item $P_{X|Y}(x | y) = P_X(x)$
			            \item $P_{Y|X}(y | x) = P_Y(y)$
		            \end{itemize}
	      \end{enumerate}
	\item The \textbf{conditional cumulative distribution functions} $F_{X|Y=y}
		      (x)$ and $F_{Y|X=x} (y)$ of a discrete distribution (where
	      $P_{X|Y}(x|y)$ is a conditional PMF for $X$ and $Y$ and opposite)
	      can be written as
	      such:
	      \begin{itemize}
		      \item ``Conditional CDF of $X$ given that $Y=y$'': $F_{X|Y=y} (x) = P(X \leq x | Y = y) = \displaystyle\sum_{a \leq x} P_{X|Y}(a | Y = y)$
		      \item ``Conditional CDF of $Y$ given that $X=x$'': $F_{Y|X=x} (y) = P(Y \leq y | X = x) = \displaystyle\sum_{a \leq y} P_{Y|X}(a | X = x)$
	      \end{itemize}
	\item The \textbf{conditional expected values} $E[X|Y = y]$ and $E[Y |X = x]$ can be calculated as such:
	      \begin{itemize}
		      \item ``Conditional expected value of $X$, given $Y$'': $\mu_{X|Y=y} = E[X | Y=y] = \displaystyle\sum_{x} x P_{X|Y}(x|y)$
		      \item ``Conditional expected value of $Y$, given $X$'': $\mu_{Y|X=x} = E[Y | X=x] = \displaystyle\sum_{y} y P_{Y|X}(y|x)$
	      \end{itemize}
	\item You can also calculate the \textbf{conditional variances}. For example:
	      \begin{itemize}
		      \item ``Conditional variance of $X$ given $Y=y$'': $\sigma_{X|Y=y}^2 = Var(X|Y=y) = E[X^2 | Y=y] - (E[X|Y=y])^2$
		      \item ``Conditional variance of $Y$ given $X=x$'': $\sigma_{Y|X=x}^2 = Var(Y|X=x) = E[Y^2 | X=x] - (E[Y|X=x])^2$
	      \end{itemize}
\end{itemize}

\subsection{Continuous Distributions} % TODO: finish
Note: for this section, whenever it says ``given $Y=y$'' (for any given random variable), you can also replace that with ``given an event $A$ has occurred.''

\begin{itemize}
	\item The \textbf{conditional probability distribution functions} $f_{X | Y}(x|y)$ and $f_{Y | X}(y|x)$ of a continuous distribution can be defined as such (where $f(x, y)$ is a joint PDF for $X$ and $Y$):
	      \begin{itemize}
		      \item ``Conditional probability distribution function of $X$ given that $Y=y$'': $f_{X | Y}(x|y) = \frac{f(x, y)}{f_Y(y)}$.
		      \item ``Conditional probability distribution function of $Y$ given that $X=x$'': $f_{Y | X}(y|x) = \frac{f(y, x)}{f_X(x)}$.
	      \end{itemize}
	\item Below are some properties of conditional PDFs:
	      \begin{enumerate}
		      \item Conditional PDFs are valid PDFs, meaning that they satisfy the properties of a PDF (see definition of PDF for Joint Distribution above).
		      \item \emph{In general}, the conditional PDF of $X$ given $Y$ does not equal the conditional distribution of $Y$ given $X$, i.e.:
		            \[
			            f_{X|Y}(x | y) \neq  f_{Y|X}(y | x)
		            \]
		      \item If $X$ and $Y$ are independent, then:
		            \begin{itemize}
			            \item $f_{X|Y}(x | y) = f_X(x)$
			            \item $f_{Y|X}(y | x) = f_Y(y)$
		            \end{itemize}
	      \end{enumerate}
	\item The \textbf{conditional cumulative distribution functions} $F_{X|Y=y}
		      (x)$ and $F_{Y|X=x} (y)$ of a discrete distribution (where
	      $f_{X|Y}(x|y)$ is a conditional PDF for $X$ and $Y$ and opposite)
	      can be written as
	      such:
	      \begin{itemize}
		      \item ``Conditional CDF of $X$ given that $Y=y$'': $F_{X|Y=y} (x) = P(X \leq x | Y = y) = \int_{-\infty}^{x} f_{X|Y}(x | y)dx$
		      \item ``Conditional CDF of $Y$ given that $X=x$'': $F_{Y|X=x} (y) = P(Y \leq y | X = x) = \int_{-\infty}^{y} f_{Y|X}(y | x)dy$
	      \end{itemize}
	\item The \textbf{conditional expected values} $E[X|Y = y]$ and $E[Y |X = x]$ can be calculated as such:
	      \begin{itemize}
		      \item ``Conditional expected value of $X$, given $Y$'': $\mu_{X|Y=y} = E[X | Y=y] \int_{x} x\cdot f_{X|Y}(x, y)dx$
		      \item ``Conditional expected value of $Y$, given $X$'': $\mu_{Y|X=x} = E[Y | X=x] \int_{y} y\cdot f_{Y|X}(y, x)dy$
	      \end{itemize}
	\item You can also calculate the \textbf{conditional variances}. For example:
	      \begin{itemize}
		      \item ``Conditional variance of $X$ given $Y=y$'': $\sigma_{X|Y=y}^2 = Var(X|Y=y) = E[X^2 | Y=y] - (E[X|Y=y])^2$
		      \item ``Conditional variance of $Y$ given $X=x$'': $\sigma_{Y|X=x}^2 = Var(Y|X=x) = E[Y^2 | X=x] - (E[Y|X=x])^2$
	      \end{itemize}
\end{itemize}


\section{Finding Joint PMFs, PDFs, and CDFs of Independent Random Variables}

\begin{itemize}
	\item  A \textbf{property of independent random variables}: if $X$ and $Y$ are independent
	      random variables, the joint PDF, PDF, CDF, Expected Values, Variance, etc. of a function
	      of $X, Y$ can be obtained by \emph{applying the same function} to the original functions'
	      PMFs, PDFs, CDF, etc.
	\item For example, assume $X$ and $Y$ are independent random variables. Then:
	      \begin{itemize}
		      \item If $P_X(x), P_Y(y)$ are PMFs for $X, Y$, respectively, then
		            the joint PMF for $XY$ is $P_XY(x, y) = P_X(x) \cdot P_Y(y)$.
		      \item If $f_X(x), f_Y(y)$ are PDFs for $X, Y$, respectively, then
		            the joint PDF for $XY$ is $f_XY(x, y) = f_X(x) \cdot f_Y(y)$.
		      \item Same for CDF, expected value, variance, etc.
	      \end{itemize}
\end{itemize}

\section{Two Useful Laws of Joint Distributions}

\subsection{Law of Total Expectation (AKA Law of Iterated Expectations)}

Let $X, Y$ be random variables in the same probability space. Then, the following holds:
\[
	E(X) = E[E(X | Y)]
\]

\subsection{Law of Total Variance (AKA Eve's Law)}
Let $X, Y$ be random variables in the same probability space and assume the
variance of $Y$ is finite (i.e. it is bound by a finite number). Then, the
following holds:

\[
	Var(Y) = E[Var(Y | X)] + Var(E[Y | X])
\]

\section{Common Multivariate (Joint) Probability Distributions}

\subsection{Multivariate Normal Probability Distribution}

\emph{Before I spend a ridiculous amount of time researching this, I will wait
	to see what is covered in lecture because the learning objectives say we only
	need to understand this in the scope of R}.

\end{document}
