\documentclass[12pt]{article}
\usepackage{amsmath}
\usepackage{amsfonts}
\usepackage{amssymb}
\usepackage{amsthm}
\usepackage{pifont}
\usepackage{fancyhdr}
\usepackage{hyperref}
\usepackage{float,graphicx}
\usepackage{framed}
\usepackage[margin=3cm, headheight=15pt]{geometry}

\pagestyle{fancyplain}
\fancypagestyle{plain}{
	\renewcommand{\headrulewidth}{0.4pt}
}

\lhead{\fancyplain{Pranav Rao}{Pranav Rao}}
\rhead{\fancyplain{Week 10 Recap}{Week 10 Recap}}

\title{Week 10 Recap}
\author{Pranav Rao}
\date{Sunday, November 26, 2023}

\begin{document}
\maketitle

\subsection{CDF Method of Finding PDF of a Function of a Random Variable}
Given that $Y = g(X)$, where $X$ is some random variable.
\begin{enumerate}
	\item Determine the possible value of $Y$ based on the values of $X$ and
	      the function $g$.
	\item Begin with the cdf $F_Y(y) = P(Y \leq y) = P(g(X) \leq y)$.
	      Express the cdf in terms of the original random variable $X$.
	\item From $P(g(X) \leq y) = P(X \leq g^{-1}(y))$.
	\item Differentiate with respect to $y$ to obtain the density $f_Y(y)$.
\end{enumerate}

\subsection{Method of Transformations to Find PDF of Function of a Random Variable}

\begin{itemize}
	\item If $g(X)$ is monotonic (i.e. either strictly increasing or decreasing over the range of $X$) so it is invertible with inverse function $X = g^{-1}(Y)$, then:
	      \[
		      f_Y(y) = f_X(g^{-1}(y)) \left|\frac{d}{dy} g^{-1}(y)\right|
	      \]
\end{itemize}

\end{document}
