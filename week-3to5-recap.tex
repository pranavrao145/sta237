\documentclass[12pt]{article}
\usepackage{amsmath}
\usepackage{amsfonts}
\usepackage{amssymb}
\usepackage{amsthm}
\usepackage{pifont}
\usepackage{fancyhdr}
\usepackage{hyperref}
\usepackage{float,graphicx}
\usepackage{framed}
\usepackage[margin=3cm, headheight=15pt]{geometry}

\pagestyle{fancyplain}
\fancypagestyle{plain}{
	\renewcommand{\headrulewidth}{0.4pt}
}

\lhead{\fancyplain{Pranav Rao}{Pranav Rao}}
\rhead{\fancyplain{STA237 Week 3 to 5 Recap}{STA237 Week 3 to 5 Recap}}

\title{STA237 Week 3 to 5 Recap}
\author{Pranav Rao}
\date{October 10, 2023}

\begin{document}
\maketitle

\section{Random Variables}
\begin{itemize}
	\item A \textbf{random variable} is a variable whose value is unknown or assigns values to each of an experiment's outcomes
	\item Two types of random variable:
	      \begin{itemize}
		      \item \textbf{Discrete:} A random variable that takes on only countable values ($X = 0, 1, 2, 3$)
		      \item \textbf{Continuous:} A random variable that can take on an infinite number of values ($X \in [1, 5]$)
	      \end{itemize}
\end{itemize}

\section{Distribution Functions}
\subsection{Definition}
\begin{itemize}
	\item A \textbf{distribution} is a function that shows the possible values
	      for a variable and how often they occur.
\end{itemize}

\subsection{PMFs vs PDFs vs CDFs}

For each distribution, there are two main functions that can be used to describe it.

\begin{itemize}
	\item The first function we can use to describe a distribution changes depending on the type of variable the
	      distribution is based on. For distributions with based on a:

	      \begin{itemize}
		      \item \textit{Discrete variable}: the first function is called a
		            \textbf{probability mass function} (PMF for short). From Wikipedia, a
		            PMF is a ``function that gives the probability that a discrete random
		            variable is exactly equal to some value.''
		      \item \textit{Continuous variable}: the first function is called a
		            \textbf{probability density function} (PDF for short, not to be
		            confused with the filetype). This function is like a PMF, but it will
		            be continuous because it is based on a continuous variable.
	      \end{itemize}

	\item The second function used to describe a distribution has the same name
	      for both type of functions. This function is called a
	      \textbf{cumulative distribution function} (CDF). The CDF is a
	      function such that, when evaluated at some $x$, it gives the
	      cumulative probability that the random variable $X$ will take a value
	      less than or equal to $X$. Depending on the type of variable, the CDF
	      is calculated differently:
	      \begin{itemize}
		      \item For a discrete random variable $X$, the CDF $F_X(x)$ is calculated as such:
		            \[
			            F_X(x) = P(X \leq x) = \displaystyle\sum_{y\leq x}P(X = y)
		            \]
		      \item For a continuous random variable $X$, the CDF $F_X(x)$ is calculated as such:
		            \[
			            F_X(x) = P(X \leq x) = \int_{-\infty}^{x}f_X(y)dy
		            \]
	      \end{itemize}
\end{itemize}

\section{Expected Values, Variance, Standard Deviation, and MGFs}

\subsection{Expected Values}
\begin{itemize}
	\item The \textbf{expected value} (also known as the \textbf{mean, expectation,
		      average, etc.}), according to Wikipedia, is informally
	      defined as ``the arithmetic mean of a large number of independently selected
	      outcomes of a random variable''. The expected value, often represented as $E[X]$ or $\mu$,
	      can be calculated as such:
	      \begin{itemize}
		      \item For a \textit{discrete} random variable $X$, given that the PMF of $X$ is $p(a)$ for some value $a$:
		            \[
			            E[X] = \mu = \displaystyle\sum_{i} a_i P(X=a_i) = \displaystyle\sum_{i} a_i p(a_i)
		            \]
		      \item For a \textit{continuous} random variable $X$, given that
		            the PDF of $X$ is $f(x)$ for some value $x$:
		            \[
			            E[X] = \mu = \int_{-\infty}^{\infty}x f(x) dx
		            \]
	      \end{itemize}
\end{itemize}


\subsection{Variance and Standard Deviation}
\begin{itemize}
	\item \textit{Intuitive definition of variance:} Intuitively, the \textbf{variance} of a random variable $X$ measures how much
	      the values of X tend to spread out or vary from the mean (average) value. Like
	      expectation, the variance also has a symbol commonly associated with it, which is $\sigma^2$.
	\item \textit{Mathematical definition of variance}: Mathematically, the \textbf{variance} of any random variable $X$ with mean $\mu$
	      is defined as:
	      \[
		      Var(X) = \sigma^2 = E[(X-\mu)^2]
	      \]
	\item  An alternative definition for variance (which is often easier to use in calculations) is:

	      \[
		      Var(X) = \sigma^2 = E[X^2] - (E[X])^2
	      \]
	\item \textit{Definition of standard deviation}: The \textbf{standard
		      deviation}, according to Wikipedia, is ``measure of the amount of
	      variation or dispersion of a set of values''. The standard deviation,
	      often represented as $\sigma$ is calculated as the square root of the
	      variance, namely:
	      \[
		      \sigma = \sqrt{\sigma^2} = \sqrt{E[(X-\mu)^2]} = \sqrt{E[X^2] - (E[X])^2}
	      \]
\end{itemize}

\subsection{Moment-Generating Functions}

\begin{itemize}
	\item A \textbf{moment} (in statistics) is a way to quantify characteristics of a given probability distribution.
	      \begin{itemize}
		      \item The first moment $E[X]$ is the \textit{expected value}.
		      \item The second moment $E[X^2]$ can be used to calculate variance (see above).
		      \item Similarly, $n$'th moment $E[X^n]$ can help provide some other useful information.
	      \end{itemize}
	\item \textit{Intuitive definition of MGF:} Intuitively, a
	      \textbf{moment-generating} function (MGF) is a function that can be
	      used to generate the function for a specific \textit{moment} of a
	      random variable $X$.
	\item \textit{Mathematical definition of MGF:} Let $X$ be a random variable
	      with CDF $F_X$. Mathematically, the moment-generating function (denoted
	      as $M_X(t)$) is calculated as:

	      \[
		      M_X(t) = E[e^{tX}]
	      \]

	      We say that the MGF exists if there exists a positive constant $a$
	      such that $M_X(s)$ is finite for all $[-a, a]$.

	\item The moment-generating function (provided the expectation exists for
	      some $t$ in a neighbourhood of $0$) is calculated differently depending
	      on the type of variable in question:
	      \begin{itemize}
		      \item For a \textit{discrete random variable} $X$ with PMF
		            $p_X$, the moment-generating function is defined as: \[
			            M_X(t) = E(e^{tX}) = \displaystyle\sum_{k} e^{tk}p_X(k)
		            \]
		      \item For a \textit{continuous random variable} $X$ with PDF
		            $f_X$, the moment-generating function is defined as: \[
			            M_X(t) = E(e^{tX}) = \int_{-\infty}^{\infty} e^{tx} f_X(x) dx
		            \]
	      \end{itemize}
	\item We can get the $n$'th moment from a moment-generating formula $M_X(t)$ by taking $n$ derivatives of the MGF and then evaluating at $0$. That is to say:
	      \[
		      E(X^n) = \frac{d^n}{dt^n} M_X(t)\Bigr\rvert_{t=0}
	      \]
\end{itemize}

\section{Common Discrete Distributions}

In statistics, there are common distributions that can be used to model certain
types of events. Memorizing these distributions can make calculating
probabilities for specific events a lot easier. This section will look at some
of the most common distributions used in this course, their pre-calculated PMF,
mean, variance, (i.e. what you would get if you tried to calculate them
yourself) and their associated R functions (see the special section on R
distribution functions later this document). NOTE: all of these random
variables are discrete because we have not explicitly covered continuous
distributions yet.

\subsection{Discrete Uniform Distribution}
A note: this distribution should be used sparingly; the continuous version of
this appears to be far more common.
\begin{itemize}
	\item \textit{Use case}: where all of the $n$ discrete outcomes are equally likely to occur
	\item \textit{Example}: rolling a fair six-sided die
	\item \textit{Notation}: $Unif(a, b)$, where $a$ is the first discrete value and $b$ is the last
	\item \textit{PMF}: $1 / n$, where $n$ is the number of possible outcomes
	\item \textit{Mean}: $\frac{a + b}{2}$, where $a$ is the first discrete value and $b$ is the last
	\item \textit{Variance}: $\frac{n^2 - 1}{12}$, where $n$ is the number of outcomes
	\item \textit{Associated R functions}: None in the standard library.
\end{itemize}
\subsection{Bernoulli Distribution}
\begin{itemize}
	\item \textit{Use case}: where there are only two possible outcomes (one success, one failure)
	\item \textit{Example}: flipping a fair coin (where $p = \frac{1}{2}$)
	\item \textit{Notation}: $Bern(p)$, where $p$ is the probability of the success occurring
	\item \textit{PMF}: $\begin{cases}
			      p \text{ if the it is the first outcome} \\
			      1 - p \text{ if the second outcome}
		      \end{cases}$
	\item \textit{Mean}: $p$, where $p$ is the probability of the success occurring
	\item \textit{Variance}: $p(1-p)$, where $p$ is the probability of the success outcome occurring
	\item \textit{Associated R functions}: \verb|dbern, pbern, qbern, rbern|
\end{itemize}

\subsection{Binomial Distribution}
\begin{itemize}
	\item \textit{Use case}: where there are $n$ Bernoulli trials run back to
	      back (still only 2 outcomes, one success, one failure)
	\item \textit{Example}: ``what is the probability of getting exactly 3
	      heads if you flip a coin 5 times?'' (where $n = 5$, $k = 3$, $p = \frac{1}{2}$)
	\item \textit{Notation}: $Bin(n,p)$ or $B(n, p)$, where $n$
	      is the number of trials and $p$ is the probability of
	      the success outcome occurring
	\item \textit{PMF}: $P(X = k) = {n \choose k} p^k (1-p)^{n-k}$, where $k$ is the number
	      of times you are hoping the success outcome will occur, $p$ is the
	      probability of the success outcome on a single trial, and $n$ is the
	      number of trials
	\item \textit{Mean}: $np$, where $p$ is the probability of the success outcome occurring and $n$ is the number of trials
	\item \textit{Variance}: $np(1-p)$, where $p$ is the probability of the success outcome occurring and $n$ is the number of trials
	\item \textit{Associated R functions}: \verb|dbinom, pbinom, qbinom, rbinom|
\end{itemize}
\subsection{Geometric Distribution}
\begin{itemize}
	\item \textit{Use case}: when you want to figure out the probability of a
	      success happening within the first $k$ independent Bernoulli trials
	\item \textit{Example}: ``what is the probability that I will get a heads
	      in the first two times I flip a fair coin?'' (where $k = 2$, $p = \frac{1}{2}$)
	\item \textit{Notation}: $Geo(p)$, where $p$ is the probability of the
	      success outcome occurring
	\item \textit{PMF}: $P(X = k) = (1-p)^{k-1}p$ where $p$ is the
	      probability of success and $k$ is the desired number of trials for the success event to occur
	\item \textit{Mean}: $\frac{1}{p}$, where $p$ is the probability of the success outcome occurring
	\item \textit{Variance}: $\frac{1-p}{p^2}$, where $p$ is the probability of the success outcome occurring
	\item \textit{Associated R functions}: \verb|dgeom, pgeom, qgeom, rgeom|
\end{itemize}
\subsection{Negative Binomial Distribution}
\begin{itemize}
	\item \textit{Use case}: when you want to figure out the probability that
	      $r$ successes appear in the first $x$ independent Bernoulli trials
	\item \textit{Example}: ``what is the probability that, if I continuously
	      flip a fair coin, I will get three heads within the first 5 trials'' (where
	      $r = 3$, $k = 5$, $p = \frac{1}{2}$)
	\item \textit{Notation}: $NB(r,p)$, where $r$ is the
	      desired number of successes and $p$ is the
	      probability of the success event occurring
	\item \textit{PMF}: $P(x = k) = {k-1\choose r-1} p^{r-1}(1-p)^{(k-1) - (r-1)}$, where $p$ is the
	      probability of the success outcome, $r$ is the desired number of
	      successes, and $k$ is the number of trials within which you want to
	      achieve $r$ successes
	\item \textit{Mean}: $\frac{r(1-p)}{p}$, where $p$ is the probability of
	      the success outcome, $r$ is the desired number of successes
	\item \textit{Variance}: $\frac{r(1-p)}{p^2}$, where $p$ is the probability of
	      the success outcome, $r$ is the desired number of successes
	\item \textit{Associated R functions}: \verb|dnbinom, pnbinom, qnbinom, rnbinom|
\end{itemize}
\subsection{Hypergeometric Distribution}
\begin{itemize}
	\item \textit{Use case}: when you want to take a sample of size $n$ from a
	      combination of 2 groups (say, a ``success'' group of size $b$ and the
	      ``failure'' group), in which are there a total of $N$ entities,
	      \underline{without replacement} (that is, these are not independent and
	      therefore not Bernoulli trials), and you want to know the probability
	      that, out of that sample, $k$ people are part of the ``success group''
	\item \textit{Example}: ``6 doctors and 19 nurses attend
	      a small conference. If all 25 names are put in the
	      hat and 5 names are randomly picked without
	      replacement, what is the probability that 4 doctors
	      and 1 nurse are picked?'' (where $N = 25$, $n = 5$, $b = 6$, and $k =
		      4$)
	\item \textit{Notation}: (I could not find a satisfactory answer for this,
	      so I'm guessing) $H(N, b, n)$, where $N$ is the total size of both
	      groups, $b$ is the size of the success group, and $n$ is the sample
	      size
	\item \textit{PMF}: $P(X = k) = \frac{{b\choose k} \cdot {N - b \choose n -
				      k}}{{N \choose n}}$, where $N$ is the total size of both groups, $b$ is
	      the size of the ``success'' group, and $k$ is the desired number of
	      elements to be drawn from the success group
	\item \textit{Mean}: $n \cdot \frac{b}{N}$, where $n$ is
	      the sample size, $b$ is the size of the success
	      group, and $N$ is the size of both groups combined
	\item \textit{Variance}: $n \cdot \frac{b}{N} \cdot
		      \frac{N - b}{N} \cdot \frac{N - n}{N - 1}$, where
	      $N$ is the total sample size, $b$ is the size of the
	      ``success'' group, and $N$ is the size of both
	      groups combined
	\item \textit{Associated R functions}: \verb|dhyper, phyper, qhyper, rhyper|
\end{itemize}
\subsection{Poisson Distribution}
\begin{itemize}
	\item \textit{Use case}: when you want to find to
	      calculate the probability of number of events
	      occurring in a fixed interval of space or time if you
	      know those events occur with a known constant mean
	      rate
	\item \textit{Example}: ``One nanogram of plutonium will
	      have an average of 2.3 radioactive decays per
	      second, and the number of decays follow a Poisson
	      distribution. What is the probability that in a
	      2-second period, there are exactly 3 radioactive
	      delays?'' (where $k=3$ and $\lambda = 2.3 \cdot 2 = 4.6$)
	\item \textit{Notation}: $Pois(\lambda)$, where
	      $\lambda$ is the rate of occurrence
	\item \textit{PMF}: $P(x = k) = \frac{\lambda^k
			      e^{-\lambda}}{k!}$
	\item \textit{Mean}: $\lambda$, where $\lambda$ is the rate of occurrence
	\item \textit{Variance}: $\lambda$, where $\lambda$ is the rate of occurrence
	\item \textit{Associated R functions}: \verb|dpois, ppois, qpois, rpois|
\end{itemize}

\section{R Distribution Functions}

\begin{itemize}
	\item R follows a similar format for all of its functions that have to do with distributions.
	\item Each distribution function has four variations, which start with four prefixes: \verb|d, p, q, r|. Here is what each one does:
	      \begin{itemize}
		      \item \verb|d| variation: returns the value of the \textbf{distribution's PMF} at the given parameters.
		      \item \verb|p| variation: returns the value of the \textbf{distribution's CDF} at the given parameters.
		      \item \verb|q| variation: returns the value of the \textbf{distribution's inverse CDF} at the given parameters.
		      \item \verb|p| variation: returns a \textbf{vector of random variables}, distributed depending on the type of distribution.
	      \end{itemize}
\end{itemize}

\end{document}
